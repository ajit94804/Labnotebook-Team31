\section{Creating a LaTeX Repository Using GitHub Desktop}

\subsection{Objective}  % This will become subsection 5.1
The objective of this lab session is to create a LaTeX repository using GitHub Desktop and to create and manage a LaTeX file within that repository. By the end of this session, you should be able to:

\begin{enumerate}
    \item Create a repository using GitHub Desktop.
    \item Create a LaTeX file within the repository.
    \item Commit and push changes to GitHub.
\end{enumerate}

\subsection{Materials Needed}  % This will become subsection 5.2
\begin{itemize}
    \item A GitHub account
    \item GitHub Desktop installed on your local machine
\end{itemize}

\subsection{Procedure}  % This will become subsection 5.3

\subsubsection{Create a GitHub Account (if needed)}  % This will become subsection 5.3.1
\begin{itemize}
    \item Go to \href{https://github.com}{GitHub} and create an account if you do not already have one.
\end{itemize}

\subsubsection{Install GitHub Desktop}  % This will become subsection 5.3.2
\begin{itemize}
    \item Download and install GitHub Desktop from \href{https://desktop.github.com}{GitHub Desktop}.
\end{itemize}

\subsubsection{Create a New Repository Using GitHub Desktop}  % This will become subsection 5.3.3
\begin{itemize}
    \item Open GitHub Desktop.
    \item Click on “File” in the menu bar and select “New repository”.
    \item Fill in the repository name (e.g., “latex-project”) and an optional description.
    \item Choose the local path where you want to save the repository.
    \item Select “Initialize this repository with a README” to add an initial README file.
    \item Click “Create repository”.
    \end{itemize}

\subsubsection{Create a LaTeX File in the Repository}  % This will become subsection 5.3.4
\begin{itemize}
    \item Open the local repository folder that was created.
    \item Create a new LaTeX file using a text editor. For example, create a file named \texttt{main.tex} with the following content:
    \begin{verbatim}
    \documentclass{article}
    \begin{document}
    \title{My First LaTeX Document}
    \author{Author Name}
    \date{\today}
    \maketitle
    \section{Introduction}
    This is a sample LaTeX document.
    \end{document}
    \end{verbatim}
    \item Save the file in the repository folder.
\end{itemize}

\subsubsection{Commit and Push Changes}  % This will become subsection 5.3.5
\begin{itemize}
    \item Open GitHub Desktop.
    \item You should see the new LaTeX file listed under “Changes”.
    \item Write a summary for your commit (e.g., “Added initial LaTeX document”).
    \item Click the “Commit to main” button.
    \item Click “Push origin” to push your changes to GitHub.
\end{itemize}

\subsubsection{Verify Repository on GitHub}  % This will become subsection 5.3.6
\begin{itemize}
    \item Go to your GitHub repository page on the web and refresh it.
    \item Verify that the LaTeX file you created is present.
\end{itemize}

\subsection{Observations}  % This will become subsection 5.4
\begin{itemize}
    \item The repository was successfully created and initialized using GitHub Desktop.
    \item A new LaTeX file was created and committed successfully.
    \item The file was pushed to GitHub and is visible in the repository.
\end{itemize}

\subsection{Conclusion}  % This will become subsection 5.5
This lab session demonstrated how to create and manage a LaTeX repository using GitHub Desktop. We successfully set up a repository, created a LaTeX file, and pushed changes. GitHub Desktop provides a straightforward interface for managing repositories and collaborating on LaTeX projects.
